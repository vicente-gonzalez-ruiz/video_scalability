\title{Video Scalability}
\author{Vicente González Ruiz}
\maketitle
\tableofcontents

\section{Quality scalability}

\begin{figure}
  \svg{graphics/quality-scalability}{1000}
  \caption{Quality scalability.}
  \label{fig:quality-scalability}
\end{figure}

Véase la Fig.~\ref{fig:quality-scalability}.
\begin{itemize}
\item
  Ideal for remote visualization environments.
\item
  By definition, $V^{[0]}:=V$.
\end{itemize}

\section{Temporal scalability}
\begin{figure}
  \svg{graphics/temporal-scalability}{1000}
  \caption{Temporal scalability.}
  \label{fig:temporal-scalability}
\end{figure}

Véase la Fig.~\ref{fig:temporal-scalability}.

\begin{itemize}
\item
  It holds that

  \begin{equation}
    V^{t}=\{V_{2^t i}\}=\{V_{2i}^{t-1}\},
  \end{equation}

  where $t$ denotes the Temporal Resolution Level (TRL).
\item
  Notice that $V:=V^{0}$.
\item
  Useful for fast random access.
\end{itemize}

%\bibliography{video-compression}
%\end{document}

\section{\href{http://inst.eecs.berkeley.edu/~ee290t/sp04/lectures/videowavelet_UCB1-3.pdf}{Spatial scalability}}

\begin{figure}
  \svg{graphics/spatial-scalability}{1000}
  \caption{Spatial scalability.}
  \label{fig:spatial-scalability}
\end{figure}

Véase la Fig.~\ref{fig:spatial-scalability}.

\begin{itemize}
\item
  Useful for low-resolution devices.
\item
  By definition, $V_i:=V_i^{<0>}$ and $V_i^{<s>}$ has a
  $\frac{Y}{2^s}\times \frac{X}{2^s}$ resolution, where $X\times Y$
  is the resolution of $V_i$.
\end{itemize}

\section{Spatial multiresolution and motion estimation/compensation}

In the LP domain, it is possible to estimate the motion at high
resolution levels using the low resolution ones. This can be used
design a video encoding system in which the motion is estimated by the
decoder instead of receiving the motion information in the
code-stream. This can increase the compression ratios (we don't need
to send the motion data), simplify the RDO (we don't need to consider
the rate of the motion data), and decouple the encoder and the decoder
in the algorithm used to estimate the motion (we can use, for example,
a lower resolution at the decoder).

\section{References}

\renewcommand{\addcontentsline}[3]{}% Remove functionality of \addcontentsline
\bibliography{image_pyramids,DWT,motion_estimation,HEVC,JPEG2000,video_compression}
